% !TeX document-id = {7a7a3ec2-16da-432a-9744-8e48006efd6e}
% !TeX encoding = UTF-8
% !TeX program = pdflatex
% !BIB program = biber

% Template Revision:
% Rev. A2 -- 2019-11-04 -- A. Varli
% Rev. B1 -- 2019-11-05 -- A. Varli

%% HINWEISE:
%% MAIN.tex ist die Hauptdatei. Hier sind sämtliche Pakete eingebunden und die allgemeine Struktur ist hier festgelegt. Im Allgemeinen müssen hier keine Änderungen vorgenommen werden.
%% In der eingebundenen Datei config.tex müssen Änderungen vorgenommen werden, die in der Datei näher erläutert sind.
%% Das Deckblatt wird mit der Datei cover/coversheet.tex eingebunden. Hier sollten keine Änderungen vorgenommen werden.
%% Für Text im Vorspann (vor der Inhaltsangabe, z.B. für Vorwort, Abstract etc.) ist die Datei frontmatter.tex vorgesehen.
%% Für den Hauptteil ist die Datei mainmatter.tex vorgesehen.
%% Das Literaturverzeichnis ist die eingebundene Datei literature.bib.
%% Für Verbesserungsvorschläge bin ich gerne offen.
%% Viel Erfolg :). Linz, im Oktober 2019, Ali Varli, a_v@gmx.net.

%% PLEASE NOTE:
%% MAIN.tex is the main file. All packages are pooled here and the general structure is defined here. In general, no changes need to be made here.
%% Changes must be made in the included file config.tex. Detailed information is in the file.
%% The cover page is included with the file cover/coversheet.tex. No changes should be made here.
%% The file frontmatter.tex is provided for text in the lead text (before the summary, i.e. for the foreword, abstract, etc.).
%% The file mainmatter.tex is intended for the main part.
%% The bibliography is the included file literature.bib.
%% I am open to suggestions for improvement.
%% Good luck :-). Linz, October 2019, Ali Varli, a_v@gmx.net.

	\documentclass[
		a4paper,
		oneside,
		onecolumn,
		openany,
		parskip=half*,
%		toc=flat,
%		chapterentrydots=true,
		table,
		11pt,
		fleqn,
%		draft
	]{scrbook}
	
	\usepackage[utf8]{inputenc}

	% !TeX encoding = UTF-8
% !TeX root = MAIN.tex

\newif\ifeng
%% HINWEISE: Hier müssen folgende Einstellungen vorgenommen werden:
%% PLEASE NOTE: Select your settings here:

%% Sprache: Falls die Dokumentensprache Deutsch ist, \engtrue mit einem %-Zeichen davor auskommentieren:
%% Language: If the document language is German, comment \engtrue with a % sign in front:
	\engtrue

%% Hier den Namen des Autors eingeben:
%% Enter the author’s name here:
	\def\name{Viktor Maximilian Loreth}

%% Hier Informationen für den rechten Block unter dem JKU-Logo eingeben, wobei die Elemente mit einem Buchstaben jeweils für die Überschrift und mit Doppelbuchstaben für den Inhalt sind. Falls Elemente nicht benötigt werden, bitte NICHT LÖSCHEN, sondern frei lassen, wie z.B. elementE bzw. elementEE.
%% Enter information here for the right block under the JKU logo, whereby the elements should have one letter for the heading and double letters for content. If the elements are not needed, DO NOT DELETE them. Simply leave them blank, such as elementE and/or elementEE.
	\def\elementA{Submitted by}
	\def\elementAA{\textbf{\name} \\ k12006268}

	\def\elementB{Submitted at}
	\def\elementBB{\textbf{Institute of Computational Perception}}

	\def\elementC{Supervisor}
	\def\elementCC{\textbf{Katharina Hoedt, PHD}}

	\def\elementE{}
	\def\elementEE{}

%% Hier Datum eingeben:
%% Enter the date:
	\def\date{\today}

%% Hier Ort eingeben:
%% Enter the location:
	\def\place{Linz}

%% Hier Titel eingeben; steht über dem K:
%% Enter the title; it appears above the K:
	\def\title{Evaluation of Interpretability Methods in Image Recognition}

%% Hier ggf. Untertitel und LVA eingeben; stehen unter dem K. Falls sie nicht benötigt werden, bitte NICHT LÖSCHEN sondern frei lassen:
%% If necessary, enter a subtitle and course here; below the K. If they are not needed, please DO NOT DELETE them. Simply leave them blank.
	\def\subtitle{Subtitle}
	\def\lva{}

%% Hier ggf. Metadaten für das PDF eingeben. Falls sie nicht benötigt werden, bitte NICHT LÖSCHEN sondern frei lassen:
%% If necessary, enter metadata for the PDF here. If it is not needed, please DO NOT DELETE them. Simply leave them blank:
	\def\pdfTitle{\title}
	\def\pdfAuthor{\name}
	\def\pdfSubject{}
	\def\pdfKeywords{}

\newif\ifthesis
%% Ab hier müssen nur Änderungen vorgenommen werden, falls es sich um eine Bachelor- oder Masterarbeit oder eine Dissertation handelt. Wenn es sich darum handelt, die Auskommentierung der folgenden Zeile aufheben:
%% Starting from this point on, only enter any changes if it is a Bachelor's or Master's thesis or a dissertation. If this is the case, uncomment the following line:
	\thesistrue

%% Hier den Typ der Arbeit eingeben (0: Bachelorarbeit, 1: Masterarbeit, 2: Dissertation, 3: Diplomarbeit):
%% Enter the type of paper here (0: Bachelor’s Thesis, 1: Master’s Thesis, 2: Dissertation, 3: Diploma Degree Thesis):
	\def\type{0}

%% Hier den angestrebten akademischen Grad eingeben:
%% Enter the desired academic degree here:
	\def\degree{Bachelor of Science}

%% Hier die Studienrichtung eingeben:
%% Enter the major here:
	\def\study{Artificial Intelligence}

	
	\usepackage[T1]{fontenc}
	\usepackage{roboto}
	\usepackage{mathpazo}
	    
	\ifeng	\usepackage[ngerman,english]{babel}
	\else	\usepackage[english,ngerman]{babel}	
	\fi
		
	\usepackage{amsmath}
	\usepackage{siunitx}
	
	\usepackage{amsmath}
    \usepackage{algorithm}
    \usepackage[noend]{algpseudocode}
    \usepackage{fancyhdr,graphicx,amsmath,amssymb}
    \include{pythonlisting}

%% Zitierweise numerisch, Literaturverzeichnis alphabetisch sortiert:
%% Citation listed numerically, bibliography listed alphabetically:
	\usepackage[backend=biber,sortlocale=auto,style=numeric-comp]{biblatex}
%% Zitierweise numerisch, Literaturverzeichnis unsortiert:
%% Citation listed numerically, bibliography unsorted:
%	\usepackage[backend=biber,sorting=none,style=numeric-comp]{biblatex}
%% Zitierweise Autor-Jahr, Literaturverzeichnis alphabetisch sortiert:
%% Citation listed by author-year, bibliography listed alphabetically:
%	\usepackage[backend=biber,style=authoryear,bibstyle=authoryear,citestyle=authoryear,maxcitenames=2]{biblatex}
	\addbibresource{literature.bib}
    \usepackage{csquotes}
    
    \usepackage[a4paper,left=30mm,right=14mm,top=27mm,bottom=10mm,includeheadfoot]{geometry}

	\usepackage{lastpage}
	\usepackage{scrlayer-scrpage}
	\pagestyle{scrheadings}
	\clearscrheadfoot
	\ifeng	\ohead*{\includegraphics[width=3cm]{cover/jkuen.png}}
	\else	\ohead*{\includegraphics[width=3cm]{cover/jkude.png}}
	\fi
	\ifoot*{\date}
	\cfoot*{\name}
	\ofoot*{\pagemark/\pageref{LastPage}}	
	\setkomafont{pageheadfoot}{\sffamily \scriptsize}
	\setkomafont{pagenumber}{\sffamily \scriptsize}

	\usepackage[onehalfspacing]{setspace}
	
	\usepackage{pdfpages}

	\usepackage{tabularx}
	\usepackage{ltxtable}
	\usepackage{booktabs}
	\usepackage{rotating}
	\usepackage{colortbl}
	\usepackage{multirow}
	
	\usepackage{xcolor}

	\usepackage{graphicx}
	\usepackage{wrapfig}

	\usepackage[section]{placeins} %\FloatBarrier

	\usepackage{float} %[H]

	\usepackage{enumitem}
		
	\usepackage{subfiles}
	
%	\usepackage[toc,lof,lot]{multitoc}

	\usepackage[
		bookmarksnumbered=true,
		pdfborder={0 0 0},
		pdfa,
		pdftitle={\pdfTitle},
		pdfauthor={\pdfAuthor},
		pdfsubject={\pdfSubject},
		pdfkeywords={\pdfKeywords}
	]{hyperref}
	
%	\setcounter{tocdepth}{3} %subsubsection
%	\setcounter{secnumdepth}{3}

    \AtBeginDocument{\AtBeginShipoutNext{\AtBeginShipoutDiscard}}
		
	\tolerance=100
	\clubpenalty=10000
	\widowpenalty=10000
	\displaywidowpenalty=10000
	
%	\addtocontents{toc}{\protect\enlargethispage{2\normalbaselineskip}}
%	\addtocontents{lof}{\protect\enlargethispage{2\normalbaselineskip}}
%	\addtocontents{lot}{\protect\enlargethispage{2\normalbaselineskip}}
	
	\addtokomafont{caption}{\small}
	\setkomafont{captionlabel}{\small\sffamily\bfseries}
	
%
%%
%%%%
%%%%%%%%
%%%%%%%%%%%%%%%%
\begin{document}
%%%%%%%%%%%%%%%%

\begin{titlepage}
{
\singlespacing
\parindent 0pt
\def\ifundefined#1{\expandafter\ifx\csname#1\endcsname\relax}
\makeatletter
\def\Huge{\@setfontsize\Huge{28pt}{28}}
\makeatother
\unitlength 1cm
\sffamily	
\small
%
%
\begin{picture}(16.6,0)
 \ifeng
  \put(11.2,0){\includegraphics[width=5.2cm]{cover/jkuen}}
 \else
  \put(11.2,0){\includegraphics[width=5.2cm]{cover/jkude}}
 \fi
 \put(12.6,-1.7){%
  \begin{minipage}[t]{3.9cm}
   \begin{flushleft}
	\ifdefined\elementA
	 {\footnotesize\elementA \vskip.1mm}
	 {\elementAA}
	 \vskip5mm
	\else
	 \relax
	\fi
	\ifdefined\elementB
	 {\footnotesize\elementB \vskip.1mm}
	 {\elementBB}
	 \vskip5mm
	\else
	 \relax
	\fi
	\ifdefined\elementC
	 {\footnotesize\elementC \vskip.1mm}
	 {\elementCC}
	 \vskip5mm
	\else
	 \relax
	\fi
	\ifdefined\elementD
	 {\footnotesize\elementD \vskip.1mm}
	 {\elementDD}
	 \vskip5mm
	\else
	 \relax
	\fi
	\ifdefined\elementE
	 {\footnotesize\elementE \vskip.1mm}
	 {\elementEE}
	 \vskip5mm
	\else
	 \relax
	\fi
	\date
   \end{flushleft}
  \end{minipage}
 }
%	
%	
 \put(12.6,-21.5){%
  \begin{minipage}[t]{3.9cm}
   {\fontseries{black}\selectfont JOHANNES KEPLER\\
  \ifeng
   UNIVERSITY
  \else
   UNIVERSIT\"{A}T
  \fi
   LINZ}\\
   Altenbergerstra{\ss}e 69\\
   4040 Linz, \"{O}sterreich\\
   www.jku.at\\
   DVR 0093696
  \end{minipage}
 }
%
%		
 \put(0,-10.2){\begin{minipage}[b]{12cm}
 \fontseries{black}\selectfont
 {\begin{flushleft}
 \Huge \expandafter\MakeUppercase\expandafter \title
 \end{flushleft}} \end{minipage}}
%	
 \put(0,-15.2){\includegraphics[width=4.4cm]{cover/arr}}
%	
 \put(0,-16.3){\begin{minipage}[t]{12cm}
  \ifthesis \Large
   \ifeng
    \ifcase\type Bachelor \or Master \or Doctoral \or Diploma \fi Thesis \vskip1mm
    {\normalsize to obtain the academic degree of} \vskip2mm
    \degree \vskip1mm
    {\normalsize in the \ifcase\type Bachelor's \or Master's \or  Doctoral \or Diploma \fi Program} \vskip2mm
   \else
    \ifcase\type Bachelorarbeit \or Masterarbeit \or Dissertation \or Diplomarbeit \fi \vskip1mm
    {\normalsize zur Erlangung des akademischen Grades} \vskip2mm
    \degree \vskip1mm
    {\normalsize im \ifcase\type Bachelorstudium \or Masterstudium \or  Doktoratsstudium \or Diplomstudium \fi} \vskip2mm
   \fi
    \study
  \else
   {\Large\lva}
   \vskip2mm
   {\Large\bfseries\subtitle} 
   \fi
 \end{minipage}}
\end{picture}
}

\end{titlepage}


%%%%%%%%%%%%
\frontmatter

% !TeX encoding = UTF-8
% !TeX root = MAIN.tex

	\ifeng \chapter*{Sworn Declaration}
	I hereby declare under oath that the submitted Diploma Thesis has been written solely by me without any third-party assistance, information other than provided sources or aids have not been used and those used have been fully documented. Sources for literal, paraphrased and cited quotes have been accurately credited.

	The submitted document here present is identical to the electronically submitted text document.

	\vskip1cm
	\place, \date

	\else \chapter*{Eidesstattliche Erklärung}
	Ich erkläre an Eides statt, dass ich die vorliegende Diplomarbeit selbstständig und ohne fremde Hilfe verfasst, andere als die angegebenen Quellen und Hilfsmittel nicht benutzt bzw. die wörtlich oder sinngemäß entnommenen Stellen als solche kenntlich gemacht habe.

	Die vorliegende Diplomarbeit ist mit dem elektronisch übermittelten Textdokument identisch.

	\vskip1cm
	\place, \date
	\fi

	\ifeng	\chapter*{Abstract}
	\fi

...

	{\let\clearpage\relax
	\ifeng	\selectlanguage{ngerman} \chapter*{Zusammenfassung}
	\else	\selectlanguage{english} \chapter*{Abstract}
	\fi

...

	\ifeng	\selectlanguage{english}
	\else 	\selectlanguage{ngerman}
	\fi}


\begin{singlespace}
\tableofcontents
\listoffigures 
%\listoftables
\end{singlespace}
	

%%%%%%%%%%%	
\mainmatter

% !TeX encoding = UTF-8
% !TeX root = MAIN.tex

%%%%%%%%%%%%%%%%
\chapter{Introduction}


%%%%%%%%%%%%%%%%
\chapter{Another Chapter}


%%%%%%%%%%%%%%%%
\chapter{Conclusion and Outlook}
 % or: Conclusion
 % or: Conclusion and Discussion
 % ...
 % whatever fits best!



%%%%%%%%%%%
\backmatter

\printbibliography

\appendix

\chapter{Appendix}

\section{MNIST-Computation}




Precision and standard deviation of the retrained models per threshold.

\begin{table}[h]
	\centering
	\begin{tabular}{|c|c|c|c|c|c|c|c|c|c|c|}
		\hline
		
		Pixels blocked & 10\% & 30\% & 50\% & 70\% & 90\% \\
		\hline
		Random Baseline & 98.64\% & 97.76\% & 96.4\% & 90.9\% & 75.5\% \\
		std & 0.42 & 0.28 & 0.39 & 0.58 & 0.65  \\
		\hline
		Integrated Gradient& 95.4\% & 91.48\% & 86.64\% & 55.52\% & 10\%  \\
		std & 0.44 & 0.83 & 2.1 & 31.64 & 0  \\
		\hline
	\end{tabular} \newline
	
	\caption{Precision and standard deviation}
	\label{tab:sclass_precision_ig}
\end{table}


\section{Food101-Computation}

Precision and standard deviation of the retrained models per threshold.

% TODO, insert results
\begin{table}[h]
	\centering
	\begin{tabular}{|c|c|c|c|c|c|c|c|c|c|c|}
		\hline
		
		Pixels blocked & 10\% & 30\% & 50\% & 70\% & 90\% \\
		\hline
		Random Baseline & \% & \% & \% & \% & \% \\
		std & & & & &  \\
		\hline
		Integrated Gradient& \% & \% & \% & \% & \%  \\
		std & & & & &  \\
		\hline
		Guided Backprop & \% & \% & \% & \% & \%  \\
		std & & & & &  \\
		\hline
	\end{tabular} \newline
	
	\caption{Precision and standard deviation}
	\label{tab:sclass_precision_ig}
\end{table}




\end{document}
